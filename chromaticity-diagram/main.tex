\documentclass[tikz]{standalone}

\usetikzlibrary{calc}
\usepackage{pgfplots}
\usepackage{pgfplotstable}

% ------------------------------------------------------------------- DATASETS

% http://steve.hollasch.net/cgindex/color/freq-rgb.html
% Spectral stimuli colors - CIE 1931
% xy coordinates from 390nm to 700nm in steps of 5nm
% 63 data points
\pgfplotstableread{spectral.dat}\SpectralLocus

% http://www.vendian.org/mncharity/dir3/blackbody/UnstableURLs/bbr_color.html
% Blackbody colors - CIE 1931
% xy coordinates from 1000K to 40000K in steps of 100K
% 391 data points
\pgfplotstableread{planckian.dat}\PlanckianLocus

% ---------------------------------------------- sRGB COLORSPACE SPECIFICATION

% http://www.color.org/sRGB.xalter

% CIE chromaticities for ITU-R BT.709 reference primaries
\def\primariesLoci{
  (0.6400,0.3300)  % R
  (0.3000,0.6000)  % G
  (0.1500,0.0600)} % B

% CIE standard illuminant D65
\def\whitepointLocus{
  (0.3127,0.3290)}

% Derived transformation matrix
\def\XYZtoRGB{
  { 3.2410}{-1.5374}{-0.4986}
  {-0.9692}{ 1.8760}{ 0.0416}
  { 0.0556}{-0.2040}{ 1.0570}}

 % Linear color to gamma corrected transform
\def\gammaCorrect{
  dup 0.0031308 le                    % if < 0.0031308
  {12.92 mul}                         % then linear transform
  {1 2.4 div exp 1.055 mul -0.055 add}% else power transform
  ifelse }

% ------------------------------------------------------------- TYPE 4 HELPERS

\def\scalarProduct#1#2#3{
  #3 mul     exch
  #2 mul add exch
  #1 mul add }

\def\applyMatrix#1#2#3#4#5#6#7#8#9{
  3 copy 3 copy
  \scalarProduct{#7}{#8}{#9} 7 1 roll
  \scalarProduct{#4}{#5}{#6} 5 1 roll
  \scalarProduct{#1}{#2}{#3} 3 1 roll }

\def\xyYtoXYZ{                        % x y Y
  3 copy 3 1 roll                     % x y Y Y x y
  add neg 1 add mul 2 index div       % x y Y Y*(-(x+y)+1)/y=Z
  4 1 roll                            % Z x y Y
  dup                                 % Z x y Y Y=Y
  5 1 roll                            % Y Z x y Y
  3 2 roll                            % Y Z y Y x
  mul exch div                        % Y Z Y*x/y=X
  3 1 roll }                          % X Y Z

\def\gammaCorrectVector{
  \gammaCorrect 3 1 roll
  \gammaCorrect 3 1 roll
  \gammaCorrect 3 1 roll}

% -------------------------------------------------------------------- DRAWING

\begin{document}
\begin{tikzpicture}

% define colorspace shading size
\def\l{360}%bp = 5in

% define colorspace shading
\pgfdeclarefunctionalshading{colorspace}
  {\pgfpointorigin}{\pgfpoint{\l bp}{\l bp}}{}{
    \l\space div exch \l\space div exch % x y   (chromaticity)
    1.0                                 % x y Y (chromaticity+luminance)
    \xyYtoXYZ                           % X Y Z (XYZ)
    \expandafter\applyMatrix\XYZtoRGB   % R G B (sRGB linear)
    \gammaCorrectVector }               % R G B (sRGB gamma corrected)

% setup all pgfplots
\pgfplotsset{
  xmin=0, xmax=1, xtick=\empty, width=360bp, 
  ymin=0, ymax=1, ytick=\empty, height=360bp,
  axis line style={draw=none,},
  scale only axis,
  clip=false,
  locus marks/.style={
    line cap=rect,
    only marks,
    mark=*,
    mark size=.5*\pgflinewidth,
    mark options={line width=0},
  },
  planckian locus/.style={
    line width=.5bp,
    black,
    locus marks,
  },
  spectral locus/.style={
    line width=1bp,
    white,
    locus marks,
  },
}

\tikzset{
  colorspace grid/.style={
    dash phase=1bp,
    dash pattern=on 2bp off 4bp,
    line width=1bp,
    line cap=rect,
    white!20!black,
  },
}

% align everything to the shading
\begin{scope} [shift={(-.5*\l bp,-.5*\l bp)}]

  \begin{scope}[scale=360bp/1cm]
    % black background and viewport
    \fill [black] (-.1,-.1) rectangle (1.1,1.1);
    % xy grid (better doing it manually, would be a mess with pgfplots)
    \draw [colorspace grid, step=.1] grid (1,1);
    % Smooth spectral locus contour for clipping
    \clip [smooth] plot file {spectral.dat} -- cycle;
    % sRGB color space
    \pgfuseshading{colorspace}
    % Reference primaries gamut
    \fill [black, even odd rule, opacity=0.5]
      rectangle +(1,1) plot coordinates {\primariesLoci} -- cycle;
    % Standard illuminant mark
    \draw [white!50!black, line width=.5bp] \whitepointLocus circle (1/360);
  \end{scope}

  \begin{axis}
    % spectral locus
    \addplot [spectral locus]
      table {\SpectralLocus}
      \foreach \n [evaluate=\n as \t using (\n-390)/5/62] in {450,460,...,650}{% nm
        node [pos=\t,sloped,allow upside down] (X) {}
        (X.center) -- ($(X.center)!3bp!(X.north)$)};
    % planckian locus
    \addplot [planckian locus]
      table {\PlanckianLocus}
      \foreach \n [evaluate=\n as \t using (\n-1)/39] in {1,2,...,12}{% kK
        node [pos=\t,sloped,allow upside down] (X) {}
        (X.center) -- ($(X.center)!3bp!(X.north)$) };
  \end{axis}

\end{scope}

\end{tikzpicture}
\end{document}
